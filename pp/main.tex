\documentclass[12pt, a4paper]{article}
\usepackage{tipa}
\usepackage{amsfonts}
\usepackage{amsmath}
\usepackage{mathrsfs}
\usepackage{amssymb}
\usepackage{latexsym, lineno, indentfirst, caption2} %caption
\usepackage[super,square,comma,numbers,sort&compress]{natbib}
\usepackage{hyperref}
% \usepackage{subcaption}
\usepackage{graphicx,color,overpic}
\usepackage[loose]{subfigure}
\usepackage[rflt]{floatflt}
\usepackage{multirow}
% \usepackage[nomarkers]{endfloat}
\usepackage{diagbox}
% \usepackage{supertabular}
% \listfiles
%%% ------------------------------
\setlength{\topmargin}{-1cm} \setlength{\oddsidemargin}{0mm}
\textwidth 16cm \textheight 24cm \parskip=6pt \subfigcapmargin=6mm
\newcommand{\newsection}[1]{\section {#1} \setcounter{equation}{0}}
\renewcommand{\thefootnote}{\fnsymbol{footnote}}
\renewcommand{\baselinestretch}{1.8}
\renewcommand{\textfraction}{0.3}
%%% ------------------------------
%\citestyle{nature}
\begin{document}
\newcommand{\lr}[1]{\langle #1 \rangle}
\newcommand{\llr}[1]{\langle \hspace{-2.5pt} \langle #1 \rangle \hspace{-2.5pt} \rangle}
%%% &=& &=& &=& &=& &=& &=& &=& &=& %%% &=& &=& &=& &=& &=& &=& &=& &=& %%% &=& &=& ===

\title{Lateral inhibition in Locust antennal lobe increase the separability of odor representations}
\author{ \small % THE AUTHORS
  Maoji Wang\footnote{\emph{E-mail address}: maoji.wang@cims.nyu.edu (M. Wang).},
  Douglas Zhou\footnote{\emph{E-mail address}: zdz@cims.nyu.edu (D. Zhou).},
  David Cai\footnote{Corresponding author. \emph{E-mail address}: cai@cims.nyu.edu (D. Cai).}
\\{\tiny{
    Department of Mathematics, MOE-LSC, and Institute of Natural Sciences, Shanghai Jiao Tong University, Shanghai, China}} \vspace{-3mm} \\{\tiny{
    Courant Institute of Mathematical Sciences and Center for Neural Science, New York University, New York, United States of America}} \vspace{-3mm} \\{\tiny{
    NYUAD Institute, New York University Abu Dhabi, Abu Dhabi, United Arab Emirates
}} } \date{} \maketitle \vspace{-10mm}

\begin{abstract} \footnotesize
  Odors are noisy multidimensional objects, which are noticed to be faster processed, broader tuned, and more uniformly distributed in representing space in Drosophila antennal lobe (AL). However, there are still exists arguments that, whether these properties are caused by vesicle-depletion from olfactory receptor neurons (ORNs) to projection neurons (PNs), or caused by presynaptic-inhibition due to the activity of inhibitory local neurons within the AL. Also, it is not clear if such transformation properties can also be observed in other animal's olfactory systems. In this work we use a %well-tested
  large-scale locust AL %point neuron
  to investigate the properties and potential consequences of olfactory processing in the locust AL. We find that, similar with the Drosophila AL, the processing are also expedited and broad tuned in locust AL. Moreover, since only the AL network is simulated in our work, we can say that, the lateral presynaptic-inhibition is sufficient to increase the odor representation's separability by optimizing their distributions in the coding space.
% s (stimulus) model  %PNonly  decoupled model
% r (response) model  %coupled intact    model
\end{abstract}

\section{Introduction}

Classifing and identifing odors are of vital importance for the survival of both individual animals and their societies. Odors are generally noisy multidimensional objects which cannot be well described by just a few variables~\citep{Wilson2006}, while they are still expected to be distinctly classified in olfactory systems. On the other hand, the main target of olfactory system is merely to classify the odor as whole objects, that is, (1) Inner structures of olfactory inputs are not as important as visual inputs or auditory inputs; and (2) The olfactory system does not have to infer or complete some other aparts from the given olfactory inputs.

Noticing various odors active various ORN sets, we can say that, odors are classfied and identified in odor detection, which is the very first stage of odor processing. It is important to ask that, why they are processed again in the antennal lobe (AL), which is a part of the deutocerebral neuropil, before reaching the higher centers, and how the antennal lobe improves the representation and identification of odors? In this study, we focus on the early stage of odor processing in locust's AL to address these questions.

Locust's olfactory pathway starts from the antennae. Whenever an odor appears, odorant molecules stimulate olfactory receptor neurons (ORNs) in antennae, and the latter stimulate neurons in AL in turn. There are two kind of neurons in AL, that is, the excitatory projection neurons (PNs) and the inhibitory local neurons (LNs). Both PNs and LNs may receieve voltage fluctuations from ORNs, while only PNs relay olfactory information to higher brain centers, i.e., the mushroom body (MB), etc. %(see Fig.~\ref{Fig:bulb} for detail).
There are two tpical dynamical oscillations observed in AL, that is, the fast oscillation and the slow pattern. The fast oscillation refers to the strong 20 Hz local field potential (LFP) oscillation that can be observed when PNs response to odors. The slow pattern relates to the fact that, the both odor specific and PN specific way of PNs' responses to odors.

Setting out from effective information transmission properties, a recent study~\citep{Bhandawat2007} described some fundamental principles of olfactory processing in the Drosophila melanogaster AL. In detail, by analyse the input and output spike trains of seval identified glomeruli, the authors of~\citep{Bhandawat2007} showed that, PN responses rise more rapidly than ORN. Moreover, the transformation in AL broadens PN tuning and produces more uniform distances between odor representations in PN coding space. In addition, portions of the odor response profile of a PN are not systematically related to their direct ORN inputs. {\bf (copied - need tobe revised)}. %, which probably indicates the presence of lateral connections between glomeruli. Finally, we show that a linear discriminator classifies odors more accurately using PN spike trains than using an equivalent number of ORN spike trains.

Due to the simple fact that odors are not easy to manipulate in experiments, we work on a carefully tested full-scale locust AL model, in which all neurons are of Hodgkin-Huxley type~\citep{Patel2009, Patel2013}. This model can well reproduce both the fast pattern and the slow pattern. In the study, we first show the concurrence of the PN spike booming and the fast oscillation. Also, we show the fast oscillation causes noise. Then, by analyzing the PN fire rates in that period, we show that, Euclidean distances between similar odors are increased, and correlation coefficient are decreased after the processing of AL. Once that established, we show that, representation distributions of the same odor usually orthogonal to the distribution of neighboring odors. Lastly, we show the increasement of odor representations separability. To end the introduction we would like to emphasize that, our result confirms Laurnet's long proposed but not yet tested hypotheses, i.e., similar odors are decorrelated and their distribution in representations space are optimized {\bf (need tobe revised)}.


\section{Results}

%A brief introduction on the experiments and stimulation.

We use a well tested point neuron model adopted from the previous studies~\citep{} to simulate the locust AL. We use 830 Hodgkin-Huxley-type cells in the model to simulate PNs, and use 300 other Hodgkin-Huxley-type cells to simulate LNs. In accordance with experiment~\citep{}, PNs in the model are excitatory and generate fast ($\sim$ 3 ms) sodium spikes, while LNs are inhibitory and generate slow calcium spikes ($\sim$ 25 ms). Once triggered, PNs would form fast cholinergic synapses (via nicotinic receptors) with other neurons, while LNs would form fast GABAergic synapses via fast GABA$_{\mbox{A}}$ receptors. Similar with previous studies~\citep{}, an extra slow inhibitory current from LNs to PNs is introduced to reproduce the slow pattern.
% The lack of anatomical or functional glomerular units containing more than one PN within the locust AL suggests that spatially uniform connectivity statistics are a reasonable assumption (Leitch and Laurent, 1996; Laurent, 1996; Wilson and Mainen, 2006).
In our model, PNs and LNs are randomly interconnected with cell-type specific connection probabilities, %. Consistent with experiment~\citep{},
and the coupling network does not evolve with time.
An odor is simulated by stimulating a subset of 120 PNs and 120 LNs in the network, and different odors are represented by different subsets of stimulated neurons. (see Methods for details)

In accordance with experiment~\citep{}, %It is verified in Fig.~\ref{Figure0:model} that,
this model exhibits 20 Hz LFP oscillations when an odor appears (see Fig.~\ref{Fig0:psd}), and its 20 Hz bandpower (integrated in the range of 15 Hz to 25 Hz) is peaking shortly after the odor is applied ($\sim$ 100 ms - 350 ms after the odor onset; see Fig.~\ref{Fig0:bandpower}). The slow GABA currents and the slow pattern dynamics would get rather strong subsequently, and as a result, the bandpower would be suppressed to a relatively low level but still much higher than the baseline bandpower. Also, it is confirmed that, the slow GABAergic current generates slow temporal patterning of PN responses (see Fig.~\ref{Fig0:slowraster} and~\ref{Fig0:slowfirerate}).

To test the response of PNs to odors, we count the average PN spiking rate in a succession of nonoverlapping timebins in the strong oscillation period, the length of each timebin is 50 ms. Then, the counted spiking numbers are divided by the length of timebins, and the change of PN firing rates vs time can be obatined (Fig.~\ref{Fig0:firerate}). Similar with the previous study~\citep{}, we say a PN responses to an odor in a timebin if it spikes in no less than 7 trials in all the 10 trials in that timebin, and the PN response rate vs time is plotted in Fig.~\ref{Fig0:response}. In accordance with experiment~\citep{}, PN spikes most actively and most reliably in the same period that the oscillation is peaking (see Fig.~\ref{Fig0:firerate} and~\ref{Fig0:response}).

\begin{figure}[htbp]%\ContinuedFloat
  \centering
    \subfigure[PSD]{\label{Fig0:psd}
    \begin{minipage}[h]{0.3\textwidth}
    \begin{overpic}[scale=0.2]{figures/fig0/psd_20Hz.eps} \end{overpic}
    \end{minipage} }
    %\hspace{0.5cm}
    \subfigure[bandpower]{\label{Fig0:bandpower}
    \begin{minipage}[h]{0.3\textwidth}
    \begin{overpic}[scale=0.2]{figures/fig0/bandpower.eps} \end{overpic}
    \end{minipage} }
  %\vspace{-15mm}
    \subfigure[raster plot]{\label{Fig0:slowraster}
    \begin{minipage}[h]{0.3\textwidth}
    \begin{overpic}[scale=0.2]{figures/fig0/v_short_line_12.eps} \end{overpic}
    \end{minipage} }
    %\hspace{0.5cm}
  \vspace{-5mm}

    \subfigure[3 PNs' firing rate]{\label{Fig0:slowfirerate}
    \begin{minipage}[h]{0.3\textwidth}
    \begin{overpic}[scale=0.2]{figures/fig0/firerate_of_a_neuron_curves.eps} \end{overpic}
    \end{minipage} }
  %\vspace{-15mm}
    \subfigure[firing rate]{\label{Fig0:firerate}
    \begin{minipage}[h]{0.3\textwidth}
    \begin{overpic}[scale=0.2]{figures/fig0/fire_rate.eps} \end{overpic}
    \end{minipage} }
    %\hspace{0.5cm}
    \subfigure[response rate]{\label{Fig0:response}
    \begin{minipage}[h]{0.3\textwidth}
    \begin{overpic}[scale=0.2]{figures/fig0/response.eps} \end{overpic}
    \end{minipage} }
  \vspace{-5mm}

%   \subfigure[synchronization]{\label{Fig0:fastsync}
%   \begin{minipage}[h]{0.4\textwidth}
%   \begin{overpic}[scale=0.25]{figures/fig0/fast_sync.eps} \end{overpic}
%   \end{minipage} }
%   \hspace{0.5cm}
%   \subfigure[correlation]{\label{Fig0:fastcorr}
%   \begin{minipage}[h]{0.4\textwidth}
%   \begin{overpic}[scale=0.25]{figures/fig0/fast_corr.eps} \end{overpic}
%   \end{minipage} }
%   \vspace{-2mm}
  \caption[Model]{\label{Figure0:model} \small Model checking. (a) is the power spectral-density analysis (PSD) applied on the LFP during stimulus presentation (single trial). (b) is the bandpower integrated in the range 15-25Hz, computed in successive nonoverlapping 50 ms timebins, and averaged over all odors and all trials. The gray span shows the region that the bandpower is peaking.
    (c) shows spiking responses of a PN to stimuli in 10 trials, in which raster odors applied from 0 s. The gray line shows the region that the stimulus is on. (d) shows the responding firing rates of 3 PNs to stimuli, averaged over 10 trials.
    (e) is the PN firing rates in sucessive nonoverlapping 50 ms timebins, averaged over all odors and trials. Odors are applied from 1 s.(f) is the responsive PN percent in successive 50 ms timebins. A PN is considered responsive to an odor, if it fire in no less than 7 trials in all the 10 trials. }
\end{figure}

Taken together, the above results indicate that our model fits experimental observations well. %mechanism modeling
In this work, we shall work on this model to show that, the lateral inhibition itself significantly improves the separability of odor representations.
---
- use a series of odors
- two modes
--- {\bf S} (stimulus) mode
--- {\bf R} (response) mode
---
Previous studies on locust's AL~\citep{} and drosophila's AL~\citep{} indicate that, the odor may be encoded in a very short period after the odor onset, since various statistical particularities can be observed from the PN spiking rates in that period. Also, the Kenyon cells in locust MB would spike in about 300 ms after an odor appears~\citep{}. We shall focus on this period (100 - 350 ms after odor onset), unless otherwise stated. Note that, the fast oscillation is also most strong in this period.

\subsection{PN activities peak eariler}
Since there is a time constraint in odor processing, it is interesting to ask if the first relay center have accelerated the odor representating process. In the {\bf S} mode, PNs response increases steadily with the increase of odor intensity before getting saturated. So we can estimate that, the PN response in the {\bf S} mode does not peak before the stimulus almost peaks. On the other hand, since responses in MB can be rapidly trigged, it would be benifit if PNs activities peak eariler in the {\bf R} mode than in the {\bf S} mode. Although this sounds not easy to achieve, a similar phenomenon has been spotted from drosophila AL~\citep{}. How about the case in locust AL?

In Fig.~\ref{Fig1:fasterCurve}, we plot peri-stimulus time histograms averaged across all odors in all PNs. As is can be seen from Fig.~\ref{Fig1:fasterCurve} that, when an odor appears, PNs' activities rise much faster in the {\bf R} mode than in the {\bf S} mode. In fact, when PNs' activities begin to decay in the {\bf R} mode, their counterparts activities in the {\bf S} mode are still growing. % When ORNs' activities peak, PNs' activities have almost decaied to the steady state.
To have a more direct comparison, we plot the peaking time of PN activities in the two modes in Fig.~\ref{Fig1:fasterDots}. Again, we can see that, PNs response in the {\bf R} mode always peak eariler than in the {\bf S} mode. In fact, in the {\bf R} mode, PNs' responses nearly always peak in the time region 100 - 350 ms after the osor appears. However, in the {\bf S} mode, their responses' peaking time are distributed from 350 ms after the odor onset to the time the odor offsets, which indicates PNs response keeps growing for $\sim$ 350 ms and keeps flat afterwards.

The reason is that, the LNs have a rather large spike threshold. As a result, LNs would rarely spike in a short period after the odor onset. Since PNs' spike threshold is much smaller and LNs barely play a role here, PNs' responses would grow dramatically due to the positive feedback mechanism in the network in that period. After that short period, the actively firing PNs and the increased odor stimulus would trigger LNs to spike, then LNs would apply fast inhibition to PNs. Therefore, PNs response would begin to decay. The whole process happens so fast that, when PNs activities begin to decay the odor stimulus is still growing. The activities of PNs and LNs would finally achieve a steady state after the stimulus is stable.

\begin{figure}[htbp]\centering
    \subfigure[response curve]{\label{Fig1:fasterCurve}
    \begin{minipage}[h]{0.4\textwidth}
    \begin{overpic}[scale=0.25]{figures/fig1/faster_curve.eps} \end{overpic}
    \end{minipage} }
    \hspace{0.5cm}
    \subfigure[peaking time]{\label{Fig1:fasterDots}
    \begin{minipage}[h]{0.4\textwidth}
    \begin{overpic}[scale=0.25]{figures/fig1/faster_dots.eps} \end{overpic}
    \end{minipage} } % \vspace{-2mm}
    \caption[Peak~eariler]{\label{Figure1:peakEariler} \small PNs in the {\bf R} mode response faster than in the {\bf S} mode.
    (a) Peak-normalized peri-stimulus time histograms (PSTHs), averaged across all odors and all PNs. Odor stimulation begins at 0 ms. (b) Comparison of PN responses peaking time in the two modes, PNs in the {\bf R} mode has a shorter latency to reach the peak than in the {\bf S} mode. ({\bf need tobe revised}) Gray lines draw boundaries around the time region 100 - 350 ms after odor onset. A few cases that peaks after odor offsets are not plotted.
    }
\end{figure}

\subsection{PNs efficience} \label{Sect:PNefficience}
%We are going to show in this section that, PNs are more efficiently used in the {\bf R} mode than in the {\bf S} mode.
%Next, we ask that, how the processing in AL changed the PN response profiles?
Next, we ask that, if the transformation in AL have also changed the coding performance. To address this problem, we plot the average response magnitudes of two types of PNs to various stimuli in~\ref{Fig2plus:resProfile}. As is can be seen from Fig.~\ref{Fig2plus:resProfile} that, PNs that response actively in the {\bf S} mode do not necessarily response actively in the {\bf R} mode.

\begin{figure}[htbp]\centering
    \subfigure[Response profiles]{\label{Fig2plus:resProfile}
    %\begin{minipage}[h]{0.75\textwidth}
    \begin{overpic}[scale=0.25]{figures/fig2plus/off2onnet_selectivity_PN12.eps} \end{overpic}
    \begin{overpic}[scale=0.25]{figures/fig2plus/off2onnet_selectivity_PN166.eps} \end{overpic}
    %\end{minipage}
    }
    \hspace{0.5cm}
    \subfigure[Selectivity]{\label{Fig2plus:selectivity}
    %\begin{minipage}[h]{0.3\textwidth}
    \begin{overpic}[scale=0.3]{figures/fig2plus/lifetimeSparseness.eps} \end{overpic}
    %\end{minipage}
    } % \vspace{-2mm}

    %% \subfigure[{\bf S} - sort]{\label{Fig2:offnetSort}
    %% \begin{minipage}[h]{0.4\textwidth}
    %% \begin{overpic}[scale=0.25]{figures/fig2/sf_avg_c100_var_1100_1350.eps} \end{overpic}
    %% \end{minipage} }
    %% \hspace{0.5cm}
    %% \subfigure[{\bf R} - sort]{\label{Fig2:onnetSort}
    %% \begin{minipage}[h]{0.4\textwidth}
    %% \begin{overpic}[scale=0.25]{figures/fig2/sf_avg_c0_var_1100_1350.eps} \end{overpic}
    %% \end{minipage} } % \vspace{-2mm}
\caption[PN~efficience]{\label{Figure2plus:PNefficience} \small PN efficience. (a) is trial averaged PN firing rate histograms, for all odors, in the time period [1100ms,1350ms]. (b) is the std. of the same odor of the offnet and {\bf R} mode, averaged over all odors, and sorted according to the mean spiking rates of all odors.}
\end{figure}


\subsection{Histogram equalization} \label{Sect:histeq}
The ``histogram equalization'' has been adopted as a specific kind of sensory transformation in the previous study~\citep{}. That is, neurons spiking rates are more uniformly distributed in the higher centers than in lower centers or in receptor neurons. As a result, representing neurons, odor space, and the setting out channel capacity are made better use in higher centers. Eventually, representations are more separable than stimuli. In this section, we tend to figure it out, whether such kind of transformation has also been engaged in locust AL.

First, we show that some PNs spike very actively in the {\bf{S}} mode, while most others are almost silence when an odor appears. As is can be seen from Fig.~\ref{Fig2:offnet}, their response histogram shows two separated peaks located at low response intensities and high response intensities, respectively. Then, we show in Fig.~\ref{Fig2:onnet} that, PNs response histogram becomes much flatter in the {\bf{R}} mode, indicating that PNs use more uniform frequencies to representating odors in the {\bf R} mode than in the {\bf S} mode.
% ...
Thus, PNs in the {\bf R} mode response less specifically than in the {\bf S} mode. In other words, PNs are more efficiently used in odor representation in the {\mf R} mode than in the {\bf S} mode.

The fact that PNs are made full used can be important since the number of PNs is kind of limited, and each KC in MB only receives inputs from a portion of PNs. %We show in this section that, similar resulut can also be observed from locust AL (see Fig.~\ref{Figure2:histflatten}). Comparing Fig.~\ref{Fig2:offnet} and~\ref{Fig2:onnet}, we can see that, when PNs are coupled, the histogram of their firing rates are much more flatter than that the case that PNs are decoupled.
On the other hand, unlike the visual stimuli which have important inner structures, the inputing odor are general recognized as a whole in AL. Thus, PNs do not need to bother to describe complex inner structures of odors, and this flatten effect would not cause a trouble. Indeed, as is can be seen latter in Sect.~\ref{Sect:classify}, the odor classfication and identification ability have in fact been improved in the {\bf R} mode than in the {\bf S} mode.

%We would also emphasize that, the histogram equalization in locust AL is mainly caused by the fast oscillation. The reason is that, (1) the slow inhibition is still rather weak in the period we are discussing, and (2) whithout the fast inhibition and the fast oscillation, there exists no more inhibitions, therefore, nearly all PNs would spike actively and the histograms would not be flatten either.

%Lastly, we plot ... Note that, the fast oscillation formed by the closed-loop in the {\bf R} mode may introduce a larger error in PN responsing spiking rates, and this is confirmed in Fig.~\ref{Fig2:offnetSort} and~\ref{Fig2:onnetSort}. Since the inactive PNs rarely spike in the {\bf S} mode, it is not surprised to see they have a smaller noise in the {\bf S} mode. However, it is quite interesting to see that, the active PNs have smaller noise in the {\bf R} mode. Since the active PNs can play an more important rol in odor representation, locust AL may benifit from this effect. %Since there are more inactive PNs than the active PNs, there would be an noise enhencement effect in the more realistic {\bf R} mode. As is shown latter that, locust can in fact benifit from this noise enhencement effect.

%% ---
%% - lateral inhibition plays an important role here
%% - from uncoupled to coupled,, (if there is no lateral inhibition, all would fire, odor representations are almost the same)
%% - fast inhibition -> fast oscillation may have played a role here (slow inhibition is still weak at that moment).
%% ---

\begin{figure}[htbp]\centering
    \subfigure[{\bf S}]{\label{Fig2:offnet}
    \begin{minipage}[h]{0.4\textwidth}
    \begin{overpic}[scale=0.25]{figures/fig2/hist_offNet.eps} \end{overpic}
    \end{minipage} }
    \hspace{0.5cm}
    \subfigure[{\bf R}]{\label{Fig2:onnet}
    \begin{minipage}[h]{0.4\textwidth}
    \begin{overpic}[scale=0.25]{figures/fig2/hist_onNet1.eps} \end{overpic}
    \end{minipage} } % \vspace{-2mm}

    %% \subfigure[{\bf S} - sort]{\label{Fig2:offnetSort}
    %% \begin{minipage}[h]{0.4\textwidth}
    %% \begin{overpic}[scale=0.25]{figures/fig2/sf_avg_c100_var_1100_1350.eps} \end{overpic}
    %% \end{minipage} }
    %% \hspace{0.5cm}
    %% \subfigure[{\bf R} - sort]{\label{Fig2:onnetSort}
    %% \begin{minipage}[h]{0.4\textwidth}
    %% \begin{overpic}[scale=0.25]{figures/fig2/sf_avg_c0_var_1100_1350.eps} \end{overpic}
    %% \end{minipage} } % \vspace{-2mm}
\caption[Hist~Flatten]{\label{Figure2:histflatten} \small The histograms of PN response magnitudes are flattened. (a) and (b) are trial averaged PN firing rate histograms, for all odors, in the time period [1100ms,1350ms]. (c) and (d) are the std. of the same odor of the {\bf S} and {\bf R} mode, averaged over all odors, and sorted according to the mean spiking rates of all odors.}
\end{figure}


\subsection{Distance enlargement among PN spiking rate vectors} \label{Sect:distance}
We plot the Euclidean distances between the PN spiking vector of each trial of an odor and the mean PN spiking vector of the other odors in Fig.~\ref{Fig3:offnet} (the {\bf S} mode) and~\ref{Fig3:onnet} (the {\bf R} mode). The diagonal elements of the two figures are the distances between the PN spiking vector of each trial of an odor to the mean spiking vector of the same odor. By comparing these diagonal elements of Fig.~\ref{Fig3:offnet} and~\ref{Fig3:onnet}, we can see that, the distance from each trial's spiking vector to its center is enlarged in the {\bf R} mode, which can be regarded an enhencement of noise.

Although noise is usually considered to be a negative factor in information processing and transmitting system, it brings some advantages here. In fact, by comparing the elements around the diagonal elements of the two figures, we can see that the distances between representing vectors are larger than their stimuli counterparts. That is, the AL representing process enlarge the distances among simlar odors (see the centering red region in Fig.~\ref{Fig3:gridcompare}). Although, the distances among rather different odors are decreased (see the blue regions in Fig.~\ref{Fig3:gridcompare}), which implies that, AL does not simplely enlarge the distance. Instead, it is in fact trying to optimize the arrangement in the representating space of different odors.

Lastly, similar result can also be obtained from Fig.~\ref{Fig3:linecompare}, in which figure we compare PN spiking vector of each trial of each odor with the mean spiking vector of the canonical odor (the first odor in this case).

\begin{figure}[htbp]\centering
    \subfigure[off Net]{\label{Fig3:offnet}
    \begin{minipage}[h]{0.4\textwidth}
    \begin{overpic}[scale=0.25]{figures/fig5n/eucd_from_shiftTrials_to_shiftCenter_c100_t1100_1350_cbar.eps} \end{overpic}
    \end{minipage} }
    \hspace{0.5cm}
    \subfigure[on net - osc]{\label{Fig3:onnet}
    \begin{minipage}[h]{0.4\textwidth}
    \begin{overpic}[scale=0.25]{figures/fig5n/eucd_from_shiftTrials_to_shiftCenter_c0_t1100_1350_cbar.eps} \end{overpic}
    \end{minipage} } % \vspace{-2mm}

    \subfigure[grid comparison]{\label{Fig3:gridcompare}
    \begin{minipage}[h]{0.4\textwidth}
    \begin{overpic}[scale=0.25]{figures/fig5n/eucd_from_shiftTrials_to_shiftCenter_0vs100_t1100_1350.eps} \end{overpic}
    \end{minipage} }
    \hspace{0.5cm}
    \subfigure[line comparison]{\label{Fig3:linecompare}
    \begin{minipage}[h]{0.4\textwidth}
    \begin{overpic}[scale=0.25]{figures/fig5n/eucd_stim_vs_resp_1100_1350_50.eps} \end{overpic}
    \end{minipage} } % \vspace{-2mm}

\caption[sprate_decorr]{\label{Figure3:distance} \small Distances from each trial's PN response vector to the mean responding vectors are tunned.}
\end{figure}


\subsection{Decorrlation of PN spiking rate vectors} \label{Sect:decorr}
The difference enlargement and decorrelation effect can also be displayed by the Pearson correction coefficient, as is plotted in Fig.~\ref{Figure4:decorr}. We can see that, the results are very similar with that of the Euclidean distance showed in Sect.~\ref{Sect:distance}.

\begin{figure}[htbp]\centering
    \subfigure[off Net]{\label{Fig4:offnet}
    \begin{minipage}[h]{0.4\textwidth}
    \begin{overpic}[scale=0.25]{figures/fig34n/ppcc_from_shiftTrials_to_shiftCenter_c100_t1100_1350_cbar.eps} \end{overpic}
    \end{minipage} }
    \hspace{0.5cm}
    \subfigure[on net - osc]{\label{Fig4:onnet}
    \begin{minipage}[h]{0.4\textwidth}
    \begin{overpic}[scale=0.25]{figures/fig34n/ppcc_from_shiftTrials_to_shiftCenter_c0_t1100_1350_cbar.eps} \end{overpic}
    \end{minipage} } % \vspace{-2mm}

    \subfigure[grid comparison]{\label{Fig4:gridcompare}
    \begin{minipage}[h]{0.4\textwidth}
    \begin{overpic}[scale=0.25]{figures/fig34n/ppcc_from_shiftTrials_to_shiftCenter_0vs100_t1100_1350.eps} \end{overpic}
    \end{minipage} }
    \hspace{0.5cm}
    \subfigure[line comparison]{\label{Fig4:linecompare}
    \begin{minipage}[h]{0.4\textwidth}
    \begin{overpic}[scale=0.25]{figures/fig34n/ppcc_stim_vs_resp_1100_1350_50.eps} \end{overpic}
    \end{minipage} } % \vspace{-2mm}

\caption[sprate_decorr]{\label{Figure4:decorr} \small the histograms of PN response magnitudes are flattened}
\end{figure}


%% \subsection{Verticality of noise distribution} \label{Sect:noiseDist}

%% From the previous sections, we have seen that, representations of neighbor odors are decorrlated, which may contribute to odor distinguish. However, the corrections from each trial to their mean are also decreased, and the distance are also increased, which may cause the odors less distinguishable. How do the AL handle this problem?

%% In this section, we show that, due to the huge number of the dimensionality of the representating space, the direction from PNs spiking rate vector of each trial to their mean of the same odor are mostly orthogonal to the direction between the mean vectors of different odors (see Fig.~\ref{Fig5:line}). And this can be illustrated by applying PCA on the PN firing rate vectors and their means (see Fig.~\ref{Fig5:vs6shift}, \ref{Fig5:vs30shift} and~\ref{Fig5:vs90shift} for detail).

%% \begin{figure}[htbp]\centering
%%     \subfigure[angle vs shifts]{\label{Fig5:line}
%%     \begin{minipage}[h]{0.4\textwidth}
%%     \begin{overpic}[scale=0.25]{figures/fig6/each_trial_vs_S0_to_others.eps} \end{overpic}
%%     \end{minipage} }
%%     \hspace{0.5cm}
%%     \subfigure[vs 6 shifts]{\label{Fig5:vs6shift}
%%     \begin{minipage}[h]{0.4\textwidth}
%%     \begin{overpic}[scale=0.25]{figures/fig6/angle_on_2D_space_S0_to_S6.eps} \end{overpic}
%%     \end{minipage} } % \vspace{-2mm}

%%     \subfigure[vs 30 shifts]{\label{Fig5:vs30shift}
%%     \begin{minipage}[h]{0.4\textwidth}
%%     \begin{overpic}[scale=0.25]{figures/fig6/angle_on_2D_space_S0_to_S30.eps} \end{overpic}
%%     \end{minipage} }
%%     \hspace{0.5cm}
%%     \subfigure[vs 90 shifts]{\label{Fig5:vs90shift}
%%     \begin{minipage}[h]{0.4\textwidth}
%%     \begin{overpic}[scale=0.25]{figures/fig6/angle_on_2D_space_S0_to_S90.eps} \end{overpic}
%%     \end{minipage} } % \vspace{-2mm}

%% \caption[Vert~dist]{\label{Figure5:vert} \small the histograms of PN response magnitudes are flattened}
%% \end{figure}


\subsection{Classify correction rate} \label{Sect:classify}
Lastly, it is important to confirm that, the larger noise really causes a larger distinguishable. Since each KC in MB only receives inputs from a portion of PNs, we use 10-50 PNs, and try to classify them according their distances to their centers defined by the mean vector...
%if the representing PN spiking vectors are really more distinguishable, esp. in the case that the noise is enlarged.

We can see that, the correction rate is always higher for the representation vectors than for the stimulation vectors (see Fig.~\ref{Figure6:classify}).

\begin{figure}[htbp]\centering
%\subfigure[classify correction rate]{\label{Fig6:rate}
\begin{minipage}[h]{0.8\textwidth}
\begin{overpic}[scale=0.35]{figures/fig7/class_corr_ratio.eps} \end{overpic}
\end{minipage}
%} \vspace{0.5cm}
\caption[classify]{\label{Figure6:classify} \small Classify correction ratio}
\end{figure}


\section{Discussion} \label{Sect:discussion}

why always orthogonal?

slow oscillation does not help.

parameters selection?

%% Odor discrimination

% PNs are divided into two sets according to their fire rate. One set mainly contributes to oscillation, and the other set is closely related to odor representation.
% Distance among odor representations in AL can be less than that among odors.
% Distance among odor representations increase slowly for very similar odors; while for odors that are significantly different with each other, distances between them increase much faster.

\section*{Acknowledgments}
This work was supported by grants from Study on Group Dynamics and Information Theory of Complex Network of Cerebral Neurons, National Natural Science Foundation of China, (No. ---); The Dynamics of Cerebral Neuronal Network, Shanghai Committee of Science and Technology, (No. ---); Features of Cerebral Neuronal Network dynamics, Shanghai Committee of Science and Technology, (No. ---) \cdots

\begin{thebibliography}{99} \scriptsize

\bibitem{Wilson2006}
Wilson RI, Mainen ZF. Early events in olfactory processing. Annu Rev Neurosci 29, 163 (2006)

\bibitem{Mazor2005}
Mazor, O., Laurent, G. Transient dynamics versus fixed points in odor representations by locust antennal lobe projection neurons. Neuron 48, 661 (2005)

\bibitem{Patel2009}
Mainak Patel, Aaditya V. Rangan, David Cai. A large-scale model of the locust antennal lobe. J Comput Neurosci 27, 553 (2009)

\bibitem{Patel2013}
Mainak J. Patel, Aaditya V. Rangan, David Cai. Coding of odors by temporal binding within a model network of the locust antennal lobe. Frontiers in Computational Neuroscience, (2013)

\bibitem{Jortner2007}
R A Jortner, S S Farivar, G Laurent. A simple connectivity scheme for sparse coding in an olfactory system. J Neurosci 27, 1659, 2007

\bibitem{Miura2012}
Keiji Miura, Zachary F. Mainen, Naoshige Uchida. Odor Representations in Olfactory Cortex: Distributed Rate Coding and Decorrelated Population Activity. Neuron 74, 1087 (2012)

\bibitem{Bhandawat2007}
V. Bhandawat, S.R. Olsen, N.W. Gouwens, M.L. Schlief, R.I. Wilson. Sensory processing in the Drosophila antennal lobe increases the reliability and separability of ensemble odor representations. Nature Neuroscience 10:1474 (2007)


% The benefits of noise in neural systems: bridging theory and experiment
% Mark D. McDonnell & Lawrence M. Ward

\end{thebibliography}

\newpage{}
\section*{Appendix} \label{Sect:appendix}
\subsection*{Model and stimuluses} \label{Sect:model}
In this study, we use the same model and stimuluses with the previous work~\citep{}, with only a few parameters adjusted. In detail, there are 830 PNs and 300 LNs in the adjusted model, and the connection probabilities among them are also adjusted with values given in Table~\ref{tab:connect_prob}. The rate constants in the concentration of receptor G proteins in slow GABA current are adjusted to $r_1 = 1.0 mM^{-1}ms^{-1}, r_2 = 0.0025 ms^{-1}, r_3 = 0.1 ms^{-1}, r_4 = 0.060 ms^{-1}$.
%% var r3, r4 float64 = 0, 0.1000, 0.0600 // !!! r4: 0.033(paper); 0.06(program) \\
%% var r1, r2 float64 = 1.0000, 0.0025 // !!! r1, r2: 0.5, 0.0013(paper); 1.0, 0.0025(pragram)

\begin{table}[htp]
\centering
\caption[connection probabilities]{connection probabilities among the neurons in AL} \label{tab:connect_prob}
\begin{tabular}{c|c c} % after \\: \ hline or \ cline {col1−col2} \cline{col3−col4 }  \cdots
\hline
\backslashbox{from}{to} & PN & LN \\ \hline
PN  & 0.010  & 0.010 \\
LN  & 0.025  & 0.025 \\ \hline
\end{tabular}
\end{table}


Odor are simulated by stimulating a set of 120 PNs and 120 LNs. In additional to the odor stimulus, PNs and LNs also accept background noises. Both odor stimulus and background noise are input in the form of Poisson spike trains with their mean rates (spike per second) and the strength of each spike ($\mu A$) given in Table~\ref{tab:inputs}.
%% Stim PN 120 (of 830), LN 120 (of 300). While in the prev work, stim PN 36 (of 90), LN 12 (of 30).
%% Each PN received current input in the form of a  with a mean rate of 3500 spikes/second and a spike strength of 0.0610 \mu A.

\begin{table}[htp]
\centering
\caption[inputs to neurons]{parameters of odor stimulus and background noise input to PNs and LNs} \label{tab:inputs}
\begin{tabular}{c|c c} % after \\: \ hline or \ cline {col1−col2} \cline{col3−col4 }  \cdots
\hline
\diagbox{source}{mean rate, strength}{target} & PNs & LNs \\ \hline
ORNs stimulus    &   35, 0.0184 &   35, 0.0176 \\
background noise & 3500, 0.0610 & 3500, 0.0001 \\ \hline
\end{tabular}
\end{table}

%% BG_input_strength_PN  float64 = 0.06100 \\
%% BG_input_strength_LN  float64 = 0.00010 \\
%% ORN_input_strength_PN float64 = 0.01840 \\
%% ORN_input_strength_LN float64 = 0.01760 \\

%% %% prev
%% %% BG_input_strength_PN  float64 = 0.06540
%% %% BG_input_strength_LN  float64 = 0.00010
%% %% ORN_input_strength_PN float64 = 0.01743
%% %% ORN_input_strength_LN float64 = 0.01667

%% \begin{table}[htop]
%% \centering
%% \caption[prev connection probabilities]{previous connection probabilities among the neurons in AL} \label{tab:connect_prob}
%% \begin{tabular}{c|c c}
%% % after \\: \ hline or \ cline {col1−col2} \cline{col3−col4 }  \cdots
%% \hline
%% \backslashbox{\(from\)}{\(to\)} & PN & LN \\ \hline
%% PN  & 0.10  & 0.10 \\
%% LN  & 0.15  & 0.25 \\ \hline
%% \end{tabular}
%% \end{table}

\subsection{Model checking} \label{Sect:model_checking}

Our model is adopted from the previous work~\citep{Patel2009, Patel2013}. In this study, we have it adjusted to the full scale, i.e., 830 PNs and 300 LNs. The network coupling probabilities and the stimulus intensities are also adjusted to reproduce the fast mode and the slow mode observed in experiments. The adjusted parameters are given in Appendix, and the reproductions of slow mode and fast mode are shown in Fig.~\ref{Fig:model}.

%%   \begin{figure}[htbp]\centering
%%   \subfigure[20 Hz peak]{\label{Fig:model_20Hz}
%%   \begin{minipage}[h]{0.4\textwidth}
%%   \begin{overpic}[scale=0.2]{figures/psd_0_0_4.eps} \end{overpic}
%%   \end{minipage} }
%%   \hspace{0.5cm}
%%   \subfigure[bandpower]{\label{Fig:model_bandpower}
%%   \begin{minipage}[h]{0.4\textwidth}
%%   \begin{overpic}[scale=0.2]{figures/sum_bandpower.eps} \end{overpic}
%%   \end{minipage} }
%%   \hspace{0.5cm}
%%   \subfigure[fire rate]{\label{Fig:model_firerate}
%%   \begin{minipage}[h]{0.4\textwidth}
%%   \begin{overpic}[scale=0.2]{figures/sum_sprate.eps} \end{overpic}
%%   \end{minipage} }
%%   \hspace{0.5cm}
%%   \subfigure[response ratio]{\label{Fig:model_respratio}
%%   \begin{minipage}[h]{0.4\textwidth}
%%   \begin{overpic}[scale=0.2]{figures/sum_response.eps} \end{overpic}
%%   \end{minipage} }
%%   \vspace{-2mm}
%%   \caption[short~Title~Here]{\label{Fig:model} \small Reproductions of slow mode and fast mode. Fig.~\ref{Fig:model_20Hz} shows a PSD of a derivated odor in a single trial. The PSD analyze analysis is applied on the averaged per PN voltage fluctuation, and the power of the very low frequency ($<5$Hz) is set to 0. Fig.~\ref{Fig:model_bandpower} gives the bandpower in {[15Hz,25Hz]} of 4 derivated odors averaged over 5 trials. The blue region indicates the max/min values of the 4 trial-averaged values corresponding to the 4 odors. Fig.~\ref{Fig:model_firerate} shows the averaged PN fire rates in successive 50ms timebins. It is also averaged (or max/min) over 4 odors and 5 trials. Fig.~\ref{Fig:model_respratio} shows the PN response ratio in successive 50ms timebins. It is also averaged (or max/min) over 4 odors, and for each odor a PN is regarded as response iff it fires in 80\% trials ($4/5$).}
%%   \end{figure}


From Fig.~\ref{Fig:model_20Hz}, we can see a 20 Hz peak; and from Fig.~\ref{Fig:model_bandpower}, we can see a peak on the appearance of an odor, which decreases afterwards and disappears when the odor is withdrawen. From Fig.~\ref{Fig:model_firerate} and \ref{Fig:model_respratio}, we can see that, both the onset bounce and offset bounce are produced. In short, most qualitative results are reproduced in our model.

%% \subsection{About the transition process} \label{Sect:transition}

Previous investigation~\citep{Mazor2005} indicates that, odor identifing may take place in a short region in the transition process shortly after stimulus presents, and PNs activities after the transition process do not contribute to the identification significantly. % Now, we are trying to locate the exact boundaries of the transition process, i.e., the region that odors may be identified.Here, we mainly focus on the stability of PNs' fire rates. % and reproducibility of PNs' fire rates.
According to previous studies and our experiences, we would mainly focus the region {[1000~ms, 1500~ms]}, that is, within 0.5 second from the odor onset. Although, some other regions are also discussed. As is can be seen from Fig.~\ref{Fig:trial_sf_corr} that, although this time region is rather short, PNs' fire rates in this region are reproducible over trials.

%%   \begin{figure}[phtb] \centering
%%   %\usepackage{float}
%%   \begin{overpic}[scale=0.3]{figures/trial_sf_corr.eps} \end{overpic} %\vspace{-2mm}
%%   \caption[qqq]{\label{Fig:trial_sf_corr} \small Correlationship of each PN's fire rate between trials in the time region {[1200~ms, 1500~ms]}. 4 odors are investigated, and for each odor the correlationship between 2 trials are plotted. All points are jittered to make the correlationship easier to observe.} % plot more figures here, for different range length and different range position. !!!!!!
%%   \end{figure}


\subsection*{Experiment} \label{Sect:experiment}
There are 4 base odors $Q_i, i=0..3$, each of which derivate 20 odors $Q_i^j, j=0..19$. Derivated odor $Q_i^j$ have $6j$ different stimulated PNs with its base odor $Q_i$. 5 trials are runned for each derivated odor, and 6.5 seconds are simulated for each trial.

\subsection*{Data processing} \label{Sect:data_proc}
 \cdots

\end{document}
